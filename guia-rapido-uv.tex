\documentclass[a4paper,9pt]{article}
\usepackage[utf8]{inputenc}
\usepackage[portuguese]{babel}
\usepackage[margin=1.2cm]{geometry}
\usepackage{multicol}
\usepackage{enumitem}
\usepackage{xcolor}
\usepackage{titlesec}
\usepackage{hyperref}
\usepackage{tcolorbox}
\usepackage{listings}

% Configurações de espaçamento
\setlength{\parindent}{0pt}
\setlength{\parskip}{0.2em}
\setlist{nosep, leftmargin=*, itemsep=0pt}

% Configurações de títulos
\titleformat{\section}{\normalsize\bfseries\color{blue!70!black}}{\thesection}{0.5em}{}
\titlespacing*{\section}{0pt}{0.4em}{0.2em}

\titleformat{\subsection}{\small\bfseries\color{blue!50!black}}{\thesubsection}{0.5em}{}
\titlespacing*{\subsection}{0pt}{0.3em}{0.15em}

\begin{document}

\begin{center}
    {\Large\bfseries Guia Rápido do uv} \\[0.2em]
    {\footnotesize Versão 0.1.0 -- Edius Ferreira}
    {\footnotesize Github @edius1987}
\date{Outubro 2025}

\end{center}

\vspace{-0.4em}

\begin{multicols}{2}

\subsection*{Instalação }
\begin{tcolorbox}[colback=gray!10, colframe=gray!50, boxrule=0.5pt, arc=2pt, left=2pt, right=2pt, top=2pt, bottom=2pt]
\texttt{\footnotesize curl -LsSf https://astral.sh/uv/install.sh | sh}
\end{tcolorbox}

\subsection*{Início Rápido}
\begin{tcolorbox}[colback=blue!5, colframe=blue!30, boxrule=0.5pt, arc=2pt, left=2pt, right=2pt, top=2pt, bottom=2pt]
{\footnotesize
\texttt{uv init myproj} \\
\texttt{uv add django requests "pandas>=2.3"} \\
\texttt{uv tree} \\
\texttt{uv run main.py}
}
\end{tcolorbox}

\subsection*{Criando projetos}
\begin{itemize}[leftmargin=0.3cm]
    \item \texttt{uv init} -- Inicializa no diretório atual
    \item \texttt{uv init myproj} -- Inicializa projeto \texttt{myproj}
    \item \texttt{uv init --app --package} -- App empacotável
    \item \texttt{uv init --lib --package} -- Biblioteca empacotável
    \item \texttt{uv init --python 3.X} -- Usa Python 3.X\textsuperscript{1}
\end{itemize}

\subsection*{Dependências}
\begin{itemize}[leftmargin=0.3cm]
    \item \texttt{uv add requests} -- Adiciona \texttt{requests}
    \item \texttt{uv add A B C} -- Adiciona múltiplas dependências
    \item \texttt{uv add "pandas>=2.3"} -- Com especificação de versão
    \item \texttt{uv add -r requirements.txt} -- Do arquivo
    \item \texttt{uv add --dev pytest} -- Dependência de desenvolvimento
    \item \texttt{uv run pytest} -- Executa pytest do projeto
    \item \texttt{uv remove django} -- Remove dependência
    \item \texttt{uv tree} -- Árvore de dependências
    \item \texttt{uv lock --upgrade} -- Atualiza versões
\end{itemize}

\subsection*{Ciclo de vida}
\begin{itemize}[leftmargin=0.3cm]
    \item \texttt{uv build} -- Constrói projeto empacotável
    \item \texttt{uv publish} -- Publica no PyPI
    \item \texttt{uv version} -- Verifica versão do projeto
    \item \texttt{uv version --bump major} -- Versão principal
    \item \texttt{uv version --bump minor --bump beta} -- Beta
    \item \texttt{uv version --bump rc} -- Release candidate
    \item \texttt{uv version --bump stable} -- Versão estável
\end{itemize}

\subsection*{Ferramentas}
\begin{itemize}[leftmargin=0.3cm]
    \item \texttt{uv tool run ruff} -- Executa em ambiente isolado
    \item \texttt{uvx ruff} -- Alias para \texttt{uv tool run}
    \item \texttt{uv tool install ruff} -- Instala globalmente
    \item \texttt{uv tool install --with dep} -- Com extras
    \item \texttt{uv tool list} -- Lista ferramentas instaladas
    \item \texttt{uv tool upgrade ruff} -- Atualiza ferramenta
    \item \texttt{uv tool upgrade --all} -- Atualiza todas
    \item \texttt{uv tool uninstall ruff} -- Desinstala
    \item \texttt{uv tool install -e .}\textsuperscript{2} -- Modo editável
\end{itemize}

\subsection*{Scripts}
\begin{itemize}[leftmargin=0.3cm]
    \item \texttt{uv init --script myscript.py} -- Inicializa script
    \item \texttt{uv init --script --python 3.X} -- Fixa versão
    \item \texttt{uv add click --script myscript.py} -- Adiciona dep
    \item \texttt{uv remove click --script myscript.py} -- Remove
    \item \texttt{uv run myscript.py} -- Executa (sem ativar venv)
    \item \texttt{uv run --python 3.X myscript.py} -- Versão específica
    \item \texttt{uv run --with click myscript.py} -- Com dependência
\end{itemize}

{\footnotesize \textbf{Dica:} Shebang \texttt{\#!/usr/bin/env -S uv run} permite \texttt{./myscript.py}}

\subsection*{Versões do Python}
\begin{itemize}[leftmargin=0.3cm]
    \item \texttt{uv python list} -- Lista versões disponíveis
    \item \texttt{uv python install 3.12} -- Instala Python 3.12
    \item \texttt{uv python uninstall 3.X} -- Desinstala
    \item \texttt{uv run python} -- Executa Python padrão
    \item \texttt{uv run --python 3.X python} -- Versão específica
    \item \texttt{uv python upgrade} -- Atualiza versões
    \item \texttt{uv python pin 3.12} -- Fixa versão no projeto
\end{itemize}

\begin{tcolorbox}[colback=green!5, colframe=green!40, boxrule=0.5pt, arc=2pt, left=2pt, right=2pt, top=2pt, bottom=2pt]
{\footnotesize \textbf{Exemplo completo:}\\
\texttt{uv init myproject} \\
\texttt{uv python pin 3.12} \\
\texttt{uv add django} \\
\texttt{uv run main.py} \\
\textit{Instalará automaticamente py3.12 e django}
}
\end{tcolorbox}

\subsection*{Para veteranos}
\begin{itemize}[leftmargin=0.3cm]
    \item \texttt{uv venv path/to/.venv} -- Cria ambiente virtual
    \item \texttt{uv pip} -- Interface do pip ⚡
\end{itemize}

\subsection*{Comandos diversos}
\begin{itemize}[leftmargin=0.3cm]
    \item \texttt{uv format} -- Formata código com Ruff
    \item \texttt{uv help cmd} -- Ajuda para comando
    \item \texttt{uv self update} -- Atualiza versão do uv
    \item \texttt{uv self version} -- Verifica versão do uv
\end{itemize}

\vfill

\hrule
\vspace{0.2em}
{\footnotesize
\textsuperscript{1}A opção \texttt{--python 3.X} é transversal a quase tudo no uv.
\textsuperscript{2}Modo editável (\texttt{-e}) evita reinstalação ao atualizar código.

\textbf{Fonte:}  \href{https://docs.astral.sh/uv/}{docs.astral.sh/uv}
}

\end{multicols}

\end{document}
